\documentclass[../body.tex]{subfiles}
\begin{document}
	В результате применения метода главных при попытке сохранить 95\% информации получается 14 векторов \eqref{tab2}, что в 4 раза меньше, чем количество признаков в исходных данных при потере информации лишь в 5\%. \\
	При этом, если рассматривать 23 компоненты, сохранятся все 99\% информации. \\
\begin{itemize}
	\item Преимущества PCA:\\
	\begin{enumerate}
		\item 	Метод позволяет облегчить работу с данными, уменьшив число факторов, требующих внимания
		\item Помогает в построении более устойчивых моделей, выполняемых быстрее, чем было бы возможно для исходных входных полей.
	\end{enumerate}
	\item Недостатки PCA:
	\begin{enumerate}
	\item Возможность непреднамеренного пренебрежения важными параметрами
	\item Использует ортогональную систему координат, что не всегда приводит к лучшим результатам
	\end{enumerate}
\end{itemize}
\end{document}